% Created 2019-12-11 三 17:12
% Intended LaTeX compiler: pdflatex
\documentclass[11pt]{article}
\usepackage[utf8]{inputenc}
\usepackage[T1]{fontenc}
\usepackage{graphicx}
\usepackage{grffile}
\usepackage{longtable}
\usepackage{wrapfig}
\usepackage{rotating}
\usepackage[normalem]{ulem}
\usepackage{amsmath}
\usepackage{textcomp}
\usepackage{amssymb}
\usepackage{capt-of}
\usepackage{hyperref}
\author{rgb-24bit}
\date{\today}
\title{MongoDB}
\hypersetup{
 pdfauthor={rgb-24bit},
 pdftitle={MongoDB},
 pdfkeywords={},
 pdfsubject={},
 pdfcreator={Emacs 26.3 (Org mode 9.2.4)}, 
 pdflang={English}}
\begin{document}

\maketitle
\tableofcontents


\section{Debian 安装}
\label{sec:org4c0b8d2}
Debian 上 mongodb 的安装还是很方便的,基本上就是一条命令的事:
\begin{verbatim}
$ apt-get install mongodb
$ mkdir -p /data/db  # Create a data directory
$ mongod             # Start service
$ mongo              # Connect to the database
\end{verbatim}

\section{概念解析}
\label{sec:org4626995}
\begin{center}
\begin{tabular}{llll}
\hline
概念 & 说明 & 例子 & RDBMS\\
\hline
database & 数据库 & db & database\\
collection & 数据集合 & db.study & table\\
document & 数据文档 & \{ "\textsubscript{id}" : ObjectId("5c52b89bb99f1ecfab25f3d9"), "name" : "rgb-24bit" \} & row\\
field & 数据字段 & \_id & column\\
index & 索引 &  & index\\
primary key & 主键 & \_id & primary key\\
\hline
\end{tabular}
\end{center}

\section{远程连接}
\label{sec:orgf66ec77}
mongodb 的远程连接配置很简单,只需要修改配置文件重启服务就足够了:
\begin{enumerate}
\item 修改配置文件 /etc/mongodb.conf 将 bind\textsubscript{ip} 修改为 0.0.0.0
\item 重启 mongodb 服务
\item 连接 mongo ip:27017,这里应该在安全组中开放该端口
\end{enumerate}

参考:\href{https://www.cnblogs.com/jinxiao-pu/p/7121307.html}{mongodb 远程连接配置 - 今孝 - 博客园}

\section{安全配置}
\label{sec:org3313718}
MongoDB 权限介绍:
\begin{enumerate}
\item MongoDB 安装时不添加任何参数,默认是没有权限验证的,登录的用户可以对数据库任意操作而且可以远程访问数据库,开启权限验证需以 --auth 参数启动
\item 在刚安装完毕的时候 MongoDB 都默认有一个 admin 数据库,此时 admin 数据库是空的,没有记录权限相关的信息。当 admin.system.users 一个用户都没有时,
即使 mongod 启动时添加了 --auth 参数,如果没有在 admin 数据库中添加用户,此时不进行任何认证还是可以做任何操作(不管是否是以 --auth 参数启动),直到在 admin.system.users 中添加了一个用户
\item MongoDB 的访问分为连接和权限验证,即使以 --auth 参数启动还是可以不使用用户名连接数据库,但是不会有任何的权限进行任何操作
\item admin 数据库中的用户名可以管理所有数据库,其他数据库中的用户只能管理其所在的数据库。
\item 在 2.4 之前版本中,用户的权限分为只读和拥有所有权限。2.4 版本的权限管理主要分为:数据库的操作权限、数据库用户的管理权限、集群的管理权限
\end{enumerate}

用户的添加方式:
\begin{verbatim}
db.addUser("test", "test")        # 默认拥有读写权限
db.addUser("test", "test", True)  # 拥有读取权限
\end{verbatim}

参考:\href{https://wooyun.js.org/drops/MongoDB安全配置.html}{MongoDB 安全配置 - zhangsan}
\end{document}
